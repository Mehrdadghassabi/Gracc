% !TeX root=../main.tex

\chapter{دیباچه}
% دستور زیر باعث عدم‌نمایش شماره صفحه در اولین صفحهٔ این فصل می‌شود.
%\thispagestyle{empty}

\section{هدف پژوهش}

هدف از این پژوهش، یافتن پاسخ برای مدار های الکتریکی به صورت خودکار %
\LTRfootnote{	automated	}
است، برای یافتن پاسخ هر مسئله به صورت خودکار نیاز است آن مسئله به صورت ریاضی مدل شود، در این مقاله برای مدل 
کردن مدار الکتریکی به صورت ریاضی از نظریه گراف
\LTRfootnote{	graph theory	}
استفاده شده است، بدین صورت که گره های مدار الکتریکی به عنوان گره های گراف در نظر گرفته شده و شاخه های مدار به عنوان یال های گراف در نظر گرفته میشود.
 
برای ایجاد یک مدل ریاضی
\LTRfootnote{	mathematical model	}
خوب از مدار های الکتریکی بایستی تمامی منطق چیره بر مدار های الکتریکی را در مدل خود بنهانیم، برای اینکار بایستی دانش های مختلفی را در هم آمیزیم،این دانش های در هم آمیخته
\LTRfootnote{	multidisciplinary science	}
عبارت اند از
دانش محاسبه،
\LTRfootnote{	Computational science	}  
 فیزیک،
\LTRfootnote{	physic	} 
  جبر خطی،
\LTRfootnote{	linear algebra	} 
   معادلات دیفرانسل،
\LTRfootnote{	differential equation	}      
     نظریه گراف،داده ساختار ها
\LTRfootnote{	data structure	}      
      و طراحی الگوریتم.
\LTRfootnote{	algorithm design	}  
\section{کاربرد پژوهش}
با پیشرفت چشمگیر قدرت محاسبه رایانه ها در قرن بیستم،
خودکار سازی پاسخ به مسائل و امکان یافتن جواب ها به صورت خودکار گسترش یافت.
از این دسته تلاش ها میتوان به مسئله دهم هیلبرت 
\footnote{	  در سال ۱۹۴۴ امیل لئون پست اثبات کرد که مسئله دهم هیلبرت تصمیم پذیر نیست بنابراین در این دسته از مسائل منظور از خودکار سازی یافتن پاسخ به معنای کمک گرفتن از قدرت محاسباتی رایانه است 	}  
اشاره کرد. یافتن خودکار پاسخ مدار های الکتریکی نیز یکیدیگر از این مسئله هاست کمااینکه امروزه نرم افزار
های زیادی مانند
 \lr{pspice}
 به وجود آمده اند که دانشجویان برق را در حل پیچیده ترین مدار ها یاری میکنند.
 
 پژوهش حاضر تلاشش بر بهبود الگوریتم های حل مدار و مدل کردن مدارالکتریکی به صورت یک داده ساختار است
 به گونه ای که بتوان از الگوریتم های ساختمان های داده و نظریه گراف در حل مدار های الکتریکی بهره جست.
\section{ساختار پایان نامه}
ساختار پایان نامه بدین گونه است که در فصل دوم و سوم به ترتیب به استفاده از دانش های فیزیک و نظریه گراف رداخته شده و سپس در فصل چهارم تلاش بر درهم آمیزی دانش های یاد شده برای مدل کردن مدار الکتریکی و یافتن پاسخ آن است در فصل پنجم به اجرای برنامه پرداخته،
و نهایتا در فصل پنجم که فصل پایانی است نتیجه گیری انجام میشود و پیشنهادهایی برای ادامه پژوهش داده میشود.


