% !TeX root=../main.tex
\chapter{مدل کردن مدار به صورت گراف}
%\thispagestyle{empty} 
\label{chap:results}
\section{پیشگفتار} 
در سه فصل پیش بر آنچه که برای مدل کردن مدار های الکتریکی به صورت گراف نیاز داریم
مرور کوتاهی شد؛ در این فصل با در آمیختن این دانش ها تلاش میشود مدار را به صورت ریاضی مدل کرده
و به صورت خودکار پاسخ آن را پیدا کنیم.
\section{گراف کیرشهف}
با در نظر گرفتن گره های مدار به عنوان راس های گراف، و شاخه های مدار به عنوان یال های گراف
میتوانیم مدار را به صورت یک گراف مدل کنیم.
در این مدل اجزای الکتریکی مدار به عنوان 
\lr{edge attribute}
در نظر گرفته میشوند.
\section{رویه یافتن جواب}
پس از مدل کردن مدار به صورت گراف، قوانین جریان و ولتاژ کیرشهف را اعمال میکنیم تا به یک
معادله ماتریسی
\footnote{
	این معادله ماتریسی در مورد مدار های دارای خازن و سلف یک معادله دیفرانسیل ماتریسی است
	در غیر این صورت یک معادله جبری ماتریسی است.
}
برسیم حل این معادله ماتریسی جریان هر شاخه مدار را به ما خواهد داد.
\cite{thulasiraman02}
\cite{Berdewad14}
در ادامه با حل چندین نمونه سعی در روشنتر شدن موضوع میکنیم.
\begin{algorithm}[ht]
	\onehalfspacing
	\caption{الگوریتم حل مدار الکتریکی
	} 
	\label{euler}
	\begin{algorithmic}[1]
		\REQUIRE
		\lr{Electrical circuit}
		\ENSURE
		\lr{Electrical circuit answer}
		\STATE
		\lr{model the circuit as a kirchhoff graph}
		\STATE
		\lr{find kirchhoff graph simple cycles}
		\STATE
		\lr{determine kvl rules from simple cycles}
		\STATE
		\lr{determine kcl rules from nodes}
		\STATE
		\lr{create matrix equation from kvl and kcl rules}
		\STATE
		\lr{solve matrix equation to find Electrical circuit answer}
		
	\end{algorithmic}
\end{algorithm}
\section{حل یک نمونه از مدار های بدون خازن و سلف}
\label{section:ex1}
به عنوان نمونه مدار شکل 
\ref{fig:fig4}
را در نظر بگیرید.
اگر نقاط
\lr{A,B,C,D,E,F,G}
را به عنوان راس های گراف انتخاب کنیم
\footnote{
	هر نقطه ای از مدار را میتوانید به دلخواه انتخاب کنید به شرطی که
	نتیجه حاصل یک گراف ساده شود.
}
گراف مدل شده
\footnote{
	به آن گراف کیرشهف نیز میگویند.
}
شکل
\ref{fig:fig5}
میشود.
\footnote{
	جهت دهی گراف را نیز به دلخواه انجام دهید؛ اگر جهت اشتباه را انتخاب کنید جریان آن شاخه صرفا منفی میشود.
}
\begin{figure}[ht]
	\centerline{\includegraphics[width=9cm]{fig4}}
	\caption{نمونه یک مدار الکتریکی }
	\label{fig:fig4}
\end{figure}

\begin{figure}[ht]
	\centerline{\includegraphics[width=9cm]{fig5}}
	\caption{گراف مدل شده شکل
		\ref{fig:fig4}	
	}
	\label{fig:fig5}
\end{figure}
پس از مدل کردن مدار به صورت گراف نوبت به یافتن دور های ساده گراف مدل شده میشود،
طبق الگوریتم
\ref{simpcyc}
تمامی دور های ساده گراف را میابیم.
این دور های ساده عبارت اند از
\lr{ABEDA}
،
\lr{BCFEB}
و
\lr{DEFD}
.
این دور های ساده هر یک مشخص کننده یک معادله
\lr{KVL}
هستند
\footnote{
	هر دور در گراف مدل شده چه ساده و چه غیرساده بیانگر یک قانون
	\lr{kvl}
	است ولی قانونی که از یک دور غیرساده بدست می آید به از لحاظ جبری
	مستقل از دور های ساده تشکیل دهنده آن نیست.
}
،بدین صورت که مجموع اختلاف پتانسیل های یک دور بایستی برابر صفر شود،
به عنوان نمونه برای دور 
\lr{ABEDA}
داریم :
\begin{equation}\label{eq:kvl_for_ABEDA}
	\Delta V_{AB} + \Delta V_{BE} + \Delta V_{ED} + \Delta V_{DA} = 0
\end{equation}
طبق عناصر موجود در هر شاخه میدانیم که
\begin{equation}\label{eq:kwneqs}
	\Delta V_{AB} = 0
	,
	\Delta V_{BE} = -40 I_3
	,
	\Delta V_{ED} = 0 
	,
	\Delta V_{DA} = - 20 + 10 I_1
	\footnote{
		دقت کنید که چون در شکل
		\ref{fig:fig5}
		جهت جریان
		\lr{i1}
		از
		\lr{A}
		به
		\lr{D}
		است بدین خاطر علامت به طور کلی
		قرینه شده است.
	}
\end{equation}
با جایگذاری معادله
\ref{eq:kwneqs}
در
\ref{eq:kvl_for_ABEDA}
به این نتیجه میرسیم:
\begin{equation}\label{eq:kvl_for_ABEDA}
	10 I_1 - 40 I_3 = 20
\end{equation}
با تکرار مراحل فوق برای دور های ساده
\lr{BCFEB}
و
\lr{DEFD}
دو معادله 
\lr{kvl}
دیگر پیدا میشود.
این دو معادله، عبارت اند از
\begin{equation}
	\begin{cases}
		10 I_7 = 0\\
		-40 I_3 + 10 I_4 = + 10
	\end{cases}\,.
\end{equation}
با یافتن تمامی قوانین
\lr{kvl}
نوبت به یافتن قوانین
\lr{kcl}
میرسد، هر راس از گراف بیانگر یک قانون 
\lr{kcl}
است ولی از 
\lr{n}
قانون موجود تنها
\lr{n-1}
قانون از لحاظ جبری مستقل
\footnote{
	استقلال جبری یک معادله از چندین معادله دیگر بدین معنی است که معادله یاد شده
	یک ترکیب خطی از دیگر معادلات نباشد.
}
اند و قانون
\lr{n}
ام،
جبرا وابسته به
\lr{n-1}
معادله پیشین است.

به عنوان نمونه معادله 
\lr{kcl}
مربوط به راس
\lr{A}
\footnote{ 
	در نظر داشته باشید که طبق شکل
	\ref{fig:fig4}
	جهت جریان 
	\lr{I0}
	و
	\lr{I1}
	اینگونه تعریف شده است،
	این جهت ها ممکن است اشتباه باشد و اگر اینطور باشد جواب بدست آمده منفی خواهد شد معنی آن این است که جریان با همین مقدار در خلاف جهت یاد شده وجود دارد.
}
برابر است با
:
\begin{equation}\label{eq:kvl_for_ABEDA}
	-I_0 -I_1 = 0
\end{equation}

با بدست آوردن معادلات
\lr{kcl}
مربوط به راس های دیگر حال بایستی اقدام به حل یک دستگاه معادلات بکنیم،
برای این مثال دستگاه بدین صورت است:
\begin{equation}
	\begin{cases}
		10 I_1 - 40 I_3 = 20\\
		10 I_7 = 0\\
		-40 I_3 + 10 I_4 = 10\\
		-I_0  -I_1 = 0\\
		I_0 -I_2 -I_3 = 0\\
		+I_2 - I_4 = 0 \\
		+I_1 - I_5 = 0 \\
		+I_3 + I_5 - I_6 = 0
	\end{cases}\,.
\end{equation}
از آنجایی که هر دستگاه معادلات خود یک معادله ماتریسی است پس داریم:
\begin{equation}
	\begin{pmatrix}
		0 & 10 & 0 & -40 & 0 & 0 & 0 & 0\\
		0 & 0 & 0 & 0 & 0 & 0 & 0 & 10\\
		0 & 0 & 0 & -40 & +10 & 0 & 0 & 0\\
		-1 & -1 & 0 & 0 & 0 & 0 & 0 & 0\\
		1 & 0 & -1 & -1 & 0 & 0 & 0 & 0\\
		0 & 0 & 1 & 0 & -1 & 0 & 0 & 0\\
		0 & 1 & 0 & 0 & 0 & -1 & 0 & 0\\
		0 & 0 & 0 & 1 & 0 & 1 & -1 & 0\\
	\end{pmatrix}
	\times
	\begin{pmatrix}
		I_0\\
		I_1\\
		I_2\\
		I_3\\
		I_4\\
		I_5\\
		I_6\\
		I_7\\
	\end{pmatrix}
	=
	\begin{pmatrix}
		20\\
		0\\
		10\\
		0\\
		0\\
		0\\
		0\\
		0\\
	\end{pmatrix}
\end{equation}

با حل معادله فوق نتیجه میشود:
\begin{equation}
	\begin{pmatrix}
		I_0\\
		I_1\\
		I_2\\
		I_3\\
		I_4\\
		I_5\\
		I_6\\
		I_7\\
	\end{pmatrix}
	=
	\begin{pmatrix}
		-\frac{2}{3}\\
		\frac{2}{3}\\
		-\frac{1}{3}\\
		-\frac{1}{3}\\
		-\frac{1}{3}\\
		\frac{2}{3}\\
		\frac{1}{3}\\
		0\\
	\end{pmatrix}
\end{equation}
بدین گونه جریان هر شاخه پیدا میشود، همانطور که پیشتر یاد شد وجود علامت منفی به این 
معنا است که جریان در خلاف جهت گمان شده وجود دارد.
گراف کیرشهف پس از یافتن جواب را در شکل
\ref{fig:fig6}
میتوانید مشاهده کنید.
\begin{figure}[ht]
	\centerline{\includegraphics[width=9cm]{fig6}}
	\caption{جواب مدار شکل
		\ref{fig:fig4}	
	}
	\label{fig:fig6}
\end{figure}

\section{حل یک نمونه از مدار
	\lr{RC} }
مدار شکل
\begin{figure}[ht]
	\centerline{\includegraphics[width=9cm]{fig10}}
	\caption{
		یک نمونه مدار
		\lr{RC}
	}
	\label{fig:fig10}
\end{figure}
\ref{fig:fig10}
و گراف مدل شده آن را در نظر بگیرید؛
\begin{figure}[ht]
	\label{fig:fig11}
	\centerline{\includegraphics[width=9cm]{fig11}}
	\caption{
		گراف مدل شده مدار
		\ref{fig:fig10}
	}
\end{figure}
میخواهیم با روندی مشابه با مثال قبل اقدام به حل آن بکنیم؛
وقتی قانون
\lr{kvl}
را برای حلقه
\lr{ABDA}
بنویسیم به معادله
\ref{eq:kvlforABDA}
میرسیم که یک معادله دیفرانسیلی است.
\footnote{در مدار های 
	\lr{RC}
	,
	\lr{RL}
	یا
	\lr{RCL}
	جریان هر شاخه در طول زمان ثابت نیست به همین خاطر در معادله
	\ref{eq:kvlforABDA}
	جریان های هر شاخه به صورت یک متغیر وابسته به زمان در نظر گرفته شده اند.
}
\begin{equation}\label{eq:kvlforABDA}
	-5I_0(t) + \frac{1}{2} \int I_2(t) \,dt = 0
\end{equation}
همچنین ممکن است با وجود خازن های دیگر در شاخه های دیگر معادلات دیفرانسیل دیگری نیز موجود باشد؛
پس بایستی اقدام به حل یک دستگاه معادلات دیفرانسیل بکنیم.

برای این مدار دستگاه معادلات در
\ref{eq:sysofode1}
آمده است.
\begin{equation}
	\label{eq:sysofode1}
	\begin{cases}
		-5I_0(t) - \frac{1}{2} \int I_1(t) \,dt = 0\\
		-5I_0(t) - 10I_2(t) - 10 = 0\\
		I_0(t) - I_3(t) + I_4(t) = 0\\
		I_2(t) - I_3(t) = 0\\
		I_1(t)  - I_4(t) = 0\\
	\end{cases}\,.
\end{equation}
که با حل آن خواهیم داشت:
\begin{equation}
	\label{eq:sysofode1}
	\begin{cases}
		I_0(t) = \frac{10}{9} + \frac{4}{9}i_1(t)\\
		I_1(t) = -\frac{40}{9} e^{-t}\\
		I_2(t) = \frac{10}{9} - \frac{5}{9}i_1(t)\\
		I_3(t) = \frac{10}{9} - \frac{5}{9}i_1(t)\\
		I_4(t) = i1\\
	\end{cases}\,.
\end{equation}

\section{جمع بندی}
برای یافتن پاسخ یک مدار الکتریکی به صورت خودکار نیازمند توسعه مدل کلی برای آن هستیم؛
در این فصل تلاش شد چگونگی مدل کردن توضیح داده شده و سپس روند اجرای این مدل بر چندین نمونه بررسی شود.