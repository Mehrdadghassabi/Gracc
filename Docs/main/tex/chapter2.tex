% !TeX root=../main.tex
\chapter{مدار الکتریکی و منطق چیره بر آن}
%\thispagestyle{empty} 
\section{پیشگفتار}
هدف از این فصل که «مدار الکتریکی و منطق چیره بر آن» نامیده شده آشکار ساختن قوانین چیره بر مدار های الکتریکی
است، قوانینی مانند قانون اهم
\LTRfootnote{	ohm law	}
و قوانین کیرشهف
\LTRfootnote{	kirchhoff laws	}
که نقش اصلی را در یافتن پاسخ مدار بازی میکنند.
\begin{itemize}
	\item
	در این فصل تلاش شده که تاریخچه کار بر روی قوانین چیره بر مدار های الکتریکی به صورت مختصر بیان شود.
	\item
	قوانین فیزیکی که در فصل های آینده مورد استفاده قرار گرفته معرفی شده اند.
	\item
	قوانین یاد شده نقش اصلی را در یافتن پاسخ مدار بازی میکنند بنابراین
	بایستی بر مدل نهایی که یک مفهوم تجریدی
	\LTRfootnote{	abstract mathematics	}
	است نیز چیره باشند.
	
\end{itemize}

\section{تاریخچه}
شاید آلساندرو ولتا را بتوان نخستین فردی نامید که در قرون معاصر بر روی مدار های الکتریکی کار کرده است،
در ابتدای قرن نوزدهم او دریافت که با متصل کردن دو کاسه نمک به وسیله نوار های فلزی میتواند جریان الکتریکی را بین آنها جاری کند.
مطالعات بر روی مدار های الکتریکی در قرن نوزدهم ادامه یافت و از دانشمندان مهمی که در این زمینه کار کردند
میتوان آندره-ماری آمپر و گئورگ زیمون اهم را نام برد.
در سال ۱۹۸۷ و در «کنفرانس عمومی وزن و اندازه‌گیری» یکای سه کمیت اختلاف پتانسیل الکتریکی، جریان الکتریکی و مقاومت الکتریکی به نام سه دانشمند یاد شده به ترتیب ولت، امپر، اهم نام گرفت.
\section{قانون اهم}
\label{formula}

نسبت اختلاف پتانسیل با جریان الکتریکی یک ماده در دمای ثابت همیشه برابر است این کمیت مقاومت الکتریکی آن ماده نامیده میشود که همانطور که پیشتر یاد شد یکای آن به افتخار گئورگ زیمون اهم، اهم نام گرفت.
\begin{equation}\label{eq:ohm}
	R=\frac{V}{I}
\end{equation}

\section{عناصر یک مدار}
عناصر الکتریکی مدار، اجزایی از مدار هستند که تغییری در انرژی مدار به وجود میاورند که خود به دو دسته عناصر کنش پذیر و عناصر کنش ناپذیر تقسیم میشوند.
از عناصر فعال میتوان منبع ولتاژ، منبع جریان و از عناصر غیر فعال میتوان مقاومت، سلف و خازن را نام برد.
\footnote{	در پروژه حاضر از تمامی اجزا مدار مانند منبع جریان پشتیبانی نمیشود برای اطلاعات بیشتر پیشنهاده را بخوانید	}

\subsection{مقاومت}
مقاومت یکی از عناصر کنش ناپذیر مدار است که باعث افت جریان در مدار میشود،
در واقع مقاومت یک مصرف کننده است که انرژی تولیدی توسط مدار را استفاده میکند.
\subsection{باتری}
باتری نیز یکی از عناصر کنش ناپذیر مدار الکتریکی است که باعث به وجود آمدن انرژی در مدار میشود.
از آنجایی که باتری غیر ایده آل
\footnote{	به باتری که مقاومت درونی نداشته باشد باتری ایده آل میگویند.	}، 
دارای مقاومت درونی است باتری با افت ولتاژ مواجه میشود در نتیجه یک غیر ایده آل مانند یک باتری ایده آل به همراه یک مقاومت رفتار میکند.
\begin{equation}\label{eq:ohm}
	V=\epsilon - {I}{r}
\end{equation}
\subsection{خازن}
خازن یا انباره همانطور که از اسمش پیداست یکی از اجزای کنش پذیر
\footnote{منظور از کنش پذیری همان 
\lr{passive}
بودن یا به عبارت دیگر غیرفعال بودن است.
}
 مدار است که انرژی را در خود ذخیره میکند
مدار هایی که شامل خازن و مقاومت هستند
\lr{RC}
نامیده میشوند که از جبر خطی پیروی میکنند.

\begin{equation}\label{eq:capcur}
	I={C}\frac{dV}{dt}
\end{equation}
\subsection{سلف}
سلف یا سیم پیچ
\footnote{در زبان انگلیسی به آن
	\lr{inductor}
	میگویند}
 یکی از عناصر کنش پذیر مدار الکتریکی است که انرژی را به صورت مغناطیسی ذخیره میکند.
مدار هایی که شامل سلف و مقاومت هستند
\lr{RL}
نامیده میشود.
مدار های
\lr{RL}
نیز مانند مدار های
\lr{RC}
از منطق جبر خطی پیروی میکند.
\footnote{مدار های شامل مقاومت، سلف و خازن را 
\lr{RLC}
می نامند که از جبر غیرخطی پیروی میکند، در پروژه حاضر از آن پشتیبانی نمیشود.	}
\begin{equation}\label{eq:capcur}
	V={L}\frac{dI}{dt}
\end{equation}

\section{قوانین کیرشهف}
قوانین کیرشهف که از دو بخش تشکیل شده خود صورتی از قانون پایستگی انرژی هستند.
\subsection{قانون جریان کیرشهف}
قانون جریان کیرشهف که به صورت مخفف
\lr{kcl}
خوانده میشود بیان میکند که مجموع جریان های ورودی و خروجی
\footnote{	جریان ورودی و خروجی در علامت متفاوت هستند معمولا جریان خروجی را منفی و جریان ورودی را مثبت در نظر میگیرند	}
 یک شاخه برابر صفر است.
 
\begin{equation}\label{eq:kcl}
	\sum_{k=1}^{n}I_k\index{k} = 0
\end{equation}
\subsection{قانون اختلاف پتانسیل کیرشهف}
قانون اختلاف پتانسیل کیرشهف که به صورت مخفف
\lr{kvl}
خوانده میشود بیان میکند که در یک حلقه بسته از مدار مجموع اختلاف پتانسیل عناصر مدار برابر با صفر است.
\begin{equation}\label{eq:kvl}
	\sum_{k=1}^{n}V_k = 0
\end{equation}

\section{جمع بندی}