% !TeX root=../main.tex
\chapter{مدار الکتریکی و منطق چیره بر آن}
%\thispagestyle{empty} 
\section{پیشگفتار}
از آنجایی که هدف ساخت یک مدل برای مدار های الکتریکی است
توضیح مفاهیم و قوانین ابتدایی مدار های الکتریکی بسیار ضروری میباشد،
زیرا این مفاهیم و قوانین بایستی به مدل ما که خود یک مفهوم مجرد
\footnote{منظور ارتباط این مفهوم با ریاضیات مجرد
\lr{abstract mathematics}
است.}
است نیز چیره باشند؛
 همچنین تاریخچه کوتاهی در ابتدای فصل برای فهم بهتر آمده است.


\section{تاریخچه و تعریف مدار}
شاید آلساندرو ولتا را بتوان نخستین فردی نامید که در قرون معاصر بر روی مدار های الکتریکی کار کرده است،
در ابتدای قرن نوزدهم او دریافت که با متصل کردن دو کاسه نمک به وسیله نوار های فلزی میتواند جریان الکتریکی را بین آنها جاری کند.
امروزه تعریف مدار الکتریکی نیز همین است مجموعه ای از عناصر الکتریکی است 
\footnote{در آزمایش ولتا کاسه های نمک نقش این عناصر را بازی میکردند
و امروزه قطعات مختلف الکتریکی}
که توسط یک رسانا به یکدیگر متصل شده اند و بین آنها جریان الکتریکی مستقیم جاری است.
\cite{Isabel20}

\section{عناصر یک مدار}
عناصر الکتریکی مدار، اجزایی از مدار هستند که تغییری در انرژی مدار به وجود میاورند که خود به دو دسته عناصر کنش پذیر و عناصر کنش ناپذیر تقسیم میشوند.
از عناصر کنش ناپذیر میتوان منبع ولتاژ، منبع جریان و از عناصر کنش پذیر میتوان مقاومت، سلف و خازن را نام برد.
در همه این عناصر سه کمیت اصلی وجود دارد که عبارت اند از
\cite{Isabel20}
\begin{itemize}
	\item
	نیرو محرکه الکتریکی : نیرویی که برای حرکت یک واحد شارژ نیاز است،واحد آن ولت می باشد.
	\item
	مقاومت الکتریکی : توانایی جلوگیری از عبور جریان الکتریکی،واحد آن اهم می باشد.
	\item
	جریان الکتریکی: نرخ تغییر شارژ در سیم، واحد آن امپر میباشد.
	
\end{itemize}


\subsection{مقاومت}
مقاومت یکی از عناصر کنش ناپذیر مدار است که باعث افت جریان در مدار میشود،
در واقع مقاومت یک مصرف کننده است که انرژی تولیدی توسط مدار را استفاده میکند.
\subsection{باتری}
باتری نیز یکی از عناصر کنش ناپذیر مدار الکتریکی است که باعث به وجود آمدن انرژی در مدار میشود.
از آنجایی که باتری غیر ایده آل
\footnote{	به باتری که مقاومت درونی نداشته باشد باتری ایده آل میگویند.	}، 
دارای مقاومت درونی است باتری با افت ولتاژ مواجه میشود در نتیجه یک غیر ایده آل مانند یک باتری ایده آل به همراه یک مقاومت رفتار میکند.
\begin{equation}\label{eq:ohm}
	V=\epsilon - {I}{r}
\end{equation}
\subsection{خازن}
خازن یا انباره همانطور که از اسمش پیداست یکی از اجزای کنش پذیر
\footnote{منظور از کنش پذیری همان 
	\lr{passive}
	بودن یا به عبارت دیگر غیرفعال بودن است.
}
مدار است که انرژی را در خود ذخیره میکند
مدار هایی که شامل خازن و مقاومت هستند
\lr{RC}
نامیده میشوند که از جبر خطی پیروی میکنند.

\begin{equation}\label{eq:capcur}
	I={C}\frac{dV}{dt}
\end{equation}
\subsection{سلف}
سلف یا سیم پیچ
\footnote{در زبان انگلیسی به آن
	\lr{inductor}
	میگویند}
یکی از عناصر کنش پذیر مدار الکتریکی است که انرژی را به صورت مغناطیسی ذخیره میکند.
مدار هایی که شامل سلف و مقاومت هستند
\lr{RL}
نامیده میشود.
مدار های
\lr{RL}
نیز مانند مدار های
\lr{RC}
از منطق جبر خطی پیروی میکند.
\begin{equation}\label{eq:slcur}
	V={L}\frac{dI}{dt}
\end{equation}

\section{مفاهیم ابتدایی}
همانطور که قبلا اشاره شد مدار مجموعه ای از قطعات الکتریکی است که به هم دیگر توسط سیم متصل اند،
این شبکه از قطعات الکتریکی دارای مفاهیم ضمنی دیگری نیز هست که عبارت اند از
\cite{Alexander13}
\begin{itemize}
	\item
	شاخه : نمایانگر یک تک قطعه الکتریکی
	\item
	گره : محل اتصال دو یا چند شاخه به یکدیگر.
	\item
	مش: هر مسیر بسته در مدار الکتریکی.
	
\end{itemize}


\section{قوانین اصلی مدار های الکتریکی}
\subsection{قانون اهم}
\label{formula}

نسبت اختلاف پتانسیل با جریان الکتریکی یک ماده در دمای ثابت همیشه برابر است این کمیت مقاومت الکتریکی آن ماده نامیده میشود که همانطور که پیشتر یاد شد یکای آن به افتخار گئورگ زیمون اهم، اهم نام گرفت.
\begin{equation}\label{eq:ohm}
	R=\frac{V}{I}
\end{equation}


\subsection{قوانین کیرشهف}
قوانین کیرشهف که از دو قانون جریان کیرشهف
و قانون اختلاف پتانسیل کیرشهف
تشکیل شده خود صورتی از قانون پایستگی انرژی هستند؛
این قوانین نقش بسیار مهمی را در یافتن پاسخ مدار الکتریکی بازی می کنند.
\subsubsection{قانون جریان کیرشهف}
قانون جریان کیرشهف که به صورت مخفف
\lr{kcl}
خوانده میشود بیان میکند که مجموع جریان های ورودی و خروجی
\footnote{	جریان ورودی و خروجی در علامت متفاوت هستند معمولا جریان خروجی را منفی و جریان ورودی را مثبت در نظر میگیرند	}
 یک شاخه برابر صفر است.
 \cite{Mahoney15}
 
\begin{equation}\label{eq:kcl}
	\sum_{k=1}^{n}I_k\index{k} = 0
\end{equation}
\subsubsection{قانون اختلاف پتانسیل کیرشهف}
قانون اختلاف پتانسیل کیرشهف که به صورت مخفف
\lr{kvl}
خوانده میشود بیان میکند که در یک حلقه بسته از مدار مجموع اختلاف پتانسیل عناصر مدار برابر با صفر است.
\cite{Mahoney15}
\begin{equation}\label{eq:kvl}
	\sum_{k=1}^{n}V_k = 0
\end{equation}

\section{جمع بندی}
مفاهیم  و قوانین فیزیکی زیر که در این فصل به آنها پرداخته شد طبیعت مدار های الکتریکی هستند و برای پیدا کردن جواب درست بایستی حتما در مدل نهایی که یک مفهوم انتزاعی است لحاظ شوند.
\begin{itemize}
	\item
	کمیت های الکتریکی مانند نیروی محرکه الکتریکی، مقاومت و جریان الکتریکی
	\item
	قوانین چیره بر قطعات اکتریکی مانند خازن، باتری، سلف و مقاومت
	\item
	مفاهیم ابتدایی مدار الکتریکی مانند شاخه، گره و مش
	\item
	قوانین ابتدایی مدار های الکتریکی مانند قوانین اهم و کیرشهف
	
\end{itemize}