% !TeX root=../main.tex
% در این فایل، عنوان پایان‌نامه، مشخصات خود، متن تقدیمی‌، ستایش، سپاس‌گزاری و چکیده پایان‌نامه را به فارسی، وارد کنید.
% توجه داشته باشید که جدول حاوی مشخصات پروژه/پایان‌نامه/رساله و همچنین، مشخصات داخل آن، به طور خودکار، درج می‌شود.
%%%%%%%%%%%%%%%%%%%%%%%%%%%%%%%%%%%%
% دانشگاه خود را وارد کنید
\university{دانشگاه اصفهان}
% دانشکده، آموزشکده و یا پژوهشکده  خود را وارد کنید
\faculty{دانشکده مهندسی کامپیوتر}
% گروه آموزشی خود را وارد کنید (در صورت نیاز)
\department{گروه مهندسی نرم افزار}
% رشته تحصیلی خود را وارد کنید
\subject{مهندسی نرم افزار}
% گرایش خود را وارد کنید

% عنوان پایان‌نامه را وارد کنید
\title{مدل کردن مدار الکتریکی به وسلیه نظریه گراف}
% نام استاد راهنما را وارد کنید
\firstsupervisor{دکتر پیمان ادیبی}
\firstsupervisorrank{استادیار}

% نام داور خود را وارد نمایید.
\internaljudge{دکتر حسین کارشناس}
\internaljudgerank{استادیار}

% نام دانشجو را وارد کنید
\name{مهرداد}
% نام خانوادگی دانشجو را وارد کنید
\surname{قصابی}
% شماره دانشجویی دانشجو را وارد کنید
\studentID{973613060}
% تاریخ پایان‌نامه را وارد کنید
\thesisdate{شهریور ۱۴۰۱}
% به صورت پیش‌فرض برای پایان‌نامه‌های کارشناسی تا دکترا به ترتیب از عبارات «پروژه»، «پایان‌نامه» و «رساله» استفاده می‌شود؛ اگر  نمی‌پسندید هر عنوانی را که مایلید در دستور زیر قرار داده و آنرا از حالت توضیح خارج کنید.
%\projectLabel{پایان‌نامه}

% به صورت پیش‌فرض برای عناوین مقاطع تحصیلی کارشناسی تا دکترا به ترتیب از عبارت «کارشناسی»، «کارشناسی ارشد» و «دکتری» استفاده می‌شود؛ اگر نمی‌پسندید هر عنوانی را که مایلید در دستور زیر قرار داده و آنرا از حالت توضیح خارج کنید.
%\degree{}
%%%%%%%%%%%%%%%%%%%%%%%%%%%%%%%%%%%%%%%%%%%%%%%%%%%%
%% پایان‌نامه خود را تقدیم کنید! %%
\dedication
{
	{\Large تقدیم به:}\\
	\begin{flushleft}{
			\huge
			مادرم که همه درد هایم را مرهم است\\
		}
	\end{flushleft}
}
%% متن قدردانی %%
%% ترجیحا با توجه به ذوق و سلیقه خود متن قدردانی را تغییر دهید.
\acknowledgement{
	سپاس و آفرین خداوندگار جان آفرین راست ، اوی که آدمی را به گوهر خرد آراست.
	
	در آغاز دستان پدر و مادر نازنینم را به پاس مهر بیکرانشان به گرمی میفشارم، 
	و از استاد راهنما خود جناب آقای دکتر پیمان ادیبی بابت زمانی که گذاشتند سپاس گزاری میکنم
	
	
	و در پایان، سپاس گزاری میکنم از همه اعضای خانواده دانشکده مهندسی کامپیوتر اصفهان به ویژه دوستانم که بهترین روز های زنگانیم را رقم زدند.
}
%%%%%%%%%%%%%%%%%%%%%%%%%%%%%%%%%%%%
%چکیده پایان‌نامه را وارد کنید
\fa-abstract{
	یک مدار الکتریکی، مجموعه ای از عناصر الکتریکی است که توسط سیم به یکدیگر متصل شده اند،
	هدف از مطالعه یک مدار الکتریکی یافتن متغیر هایی مانند جریان الکتریکی هر عنصر و به طور کلی منطق چیره بر کل مدار است که اصطلاحا به آن پاسخ مدار میگویند.
	
	دانش محاسبه دانشی است که به یافتن خودکار پاسخ مسائل می پردازد، برای یافتن پاسخ یک مدار الکتریکی به صورت خودکار، نخست بایستی مسئله به صورت ریاضی مدل شود،
	در این مقاله تلاش شده است که با استفاده از نظریه گراف، مدار الکتریکی را به صورت ریاضی مدل شده و سپس به کمک  الگوریتم های گراف و جبرخطی پاسخ آن به صورت خودکار یافت گردد.
	
}
% کلمات کلیدی پایان‌نامه را وارد کنید
\keywords{مدار الکتریکی،نظریه گراف،دانش محاسبات،جبر خطی}
% انتهای وارد کردن فیلد‌ها
%%%%%%%%%%%%%%%%%%%%%%%%%%%%%%%%%%%%%%%%%%%%%%%%%%%%%%